% classe du document
% la propriété hidelinks enlève les carrés rouges des éléments clickables
\documentclass[a4paper,hidelinks]{article}
% marges document
\usepackage[left=4cm,top=3cm,right=4cm,bottom=3cm]{geometry}

%%%%%%%%%%%%%%%%%%%%
% paquets utilisés %
%%%%%%%%%%%%%%%%%%%%

% windows : latin1. mac : applenew
\usepackage[utf8]{inputenc}
% \usepackage[french]{babel}
% \usepackage[french]{varioref}
\usepackage[nottoc,notlot]{tocbibind}
\setcounter{tocdepth}{3}
%\usepackage{fullpage}
\usepackage{amsmath}
\usepackage{amsfonts}
\usepackage{amssymb}
\usepackage{color}
\usepackage{titlesec}
\usepackage{graphicx}
\usepackage{float}
\usepackage{subfig}
%\usepackage{indentfirst}
%\usepackage{tocloft} % produit erreur sur la définition de la profondeur de champ
%\usepackage{url}
\usepackage{hyperref}
\usepackage{wrapfig}
\usepackage{minted}
\usepackage{verbatim}
\usepackage{xcolor}
\usepackage{tikz}
% \usepackage[autostyle=true]{csquotes}
\usepackage{wrapfig}
\usepackage{pdfpages}
\usepackage{eurosym}
\usepackage{tikz}
\usetikzlibrary{calc}

%%%%%%%%%%%%%%%%%%%%%%%%%%%%%%%%%%%%%%%%%%%%%%%%%%%%%%%%%
% modification noms tables, listes, indexes, glossaires %
%%%%%%%%%%%%%%%%%%%%%%%%%%%%%%%%%%%%%%%%%%%%%%%%%%%%%%%%%

%\renewcommand*\contentsname{Table des matières}
%\renewcommand*\listfigurename{Table des figures}

%%%%%%%%%%%%%%%%%%%%%%%%%%%%%%%%%%%%%
% définition de nouvelles commandes %
%%%%%%%%%%%%%%%%%%%%%%%%%%%%%%%%%%%%%
% à la puissance
\newcommand{\ts}{\textsuperscript}
%intégration logo Crédit Agricole dans page de garde
\newcommand{\logo}[5]{%
	\begin{tikzpicture}[remember picture,overlay]%
	\node[anchor=north east, minimum width=2.8] at ($(current page.north east)-(#1,#2)$) {%
		\includegraphics[width=#4,height=#5]{#3}%
	};%
	\end{tikzpicture}%
}%

%%%%%%%%%%%%%%%%%%%%%%%%%%%%%%%%%%%%%
% paramétrage des différents titres %
%%%%%%%%%%%%%%%%%%%%%%%%%%%%%%%%%%%%%

% définition des couleurs pour les titres
%\definecolor{vert}{rgb}{0,0.5,0}
%\definecolor{vertclair}{rgb}{0.85,1,0.85}
%\definecolor{redclair}{rgb}{1,0.9,0.75}

% numérotation et couleur des titres
%\renewcommand{\thesection}{\textcolor{red}}
%\renewcommand{\thesubsection}{\textcolor{blue}}
%\renewcommand{\thesubsubsection}{\textcolor{vert}}