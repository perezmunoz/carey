%!TEX root = 0_architecture_rapport.tex

% subsection title
\section*{Panel data. What predicts the players' salaries in the past 10 years? (20 points)}
\addcontentsline{toc}{section}{Panel data. What predicts the players' salaries in the past 10 years? (20 points)}
\label{subsec:3Q3}

In this question, we used our data on player performances during the past ten years to better understand the factors that determine players' salaries. Classical linear regression techniques cannot necessarily take advantage of time-dependent variables measured on different entities. Therefore, a panel regression is more suited for this problem.


\paragraph{Time-varying predictors}It is quite difficult to know in real life exactly how many variables are correlated, and if they are, how important will be their bias on the regression result. Therefore, our first idea is to keep only the following information about players for each season: age, position, games played, points scored, 3-points average, number of assists, number of steals, total rebounds, Win-Loss percentage of team during season; and some dummy variables to distinguish players that played exceptionally well during each season: championship winner member, championship runner-up member and best points, rebounds, assists, win-shares performance during the whole championship.

\paragraph{Time-invariant predictors}All these variables are time-invariant. Indeed, as we saw in question 2, height and weight have little correlation with players' salaries; therefore we decided not to include them in this panel data. Moreover, fixed effect model couldn't take advantage of these variables. The dummy variables we introduced aim at separating exceptional players from others, since we think that salaries do not necessarily depend linearly on performances but also on player's fame, which can be less rational.

\paragraph{Regression}The data we use consists in players that have been active in the past 10 seasons (i.e. from 2004-2005 to 2014-2015) for at least 8 seasons. This limitation aims at reducing the bias resulting from unbalanced data. After removing players who do not match these criteria, we have still 202 players for 1907 observations.
The result of the panel regression is shown here:

\begin{minted}[frame=single,linenos,mathescape,fontsize=\small]{r}
# Panel regression
Coefficients :
                     	Estimate  Std. Error t-value  Pr(>|t|)   
Age                     409870.87	 30376.82 13.4929 < 2.2e-16 ***
G                   	-33248.53  	  6448.52 -5.1560  2.82e-07 ***
PTS                   	3003.38  	   486.98  6.1674  8.67e-10 ***
X3P                   	4709.47 	  3256.09  1.4464 0.1482633	
AST                   	3642.69 	  1526.60  2.3861 0.0171358 * 
STL                  	-9675.03 	  6328.84 -1.5287 0.1265214	
PosC                    863717.09  2109942.06  0.4094 0.6823305	
PosPF                   784927.66  2074082.68  0.3784 0.7051471	
PosPG                  1506901.02  2027284.45  0.7433 0.4573975	
PosSF                   138135.30  2009546.42  0.0687 0.9452051	
PosSG                  1390703.09  2002512.41  0.6945 0.4874774	
WL_percentage         -1391715.89   592015.18 -2.3508 0.0188471 * 
Champion               -194477.12   465851.92 -0.4175 0.6763911	
RunnerUp             	11731.55    415405.61  0.0282 0.9774731	
TopPerformerPoints      831576.87  1193764.94  0.6966 0.4861490	
TopPerformerAssists   -1695243.10  1259194.51 -1.3463 0.1783893	
TopPerformerRebounds  -1306341.45  1245226.75 -1.0491 0.2942919	
TopPerformerWinShares  4941261.14  1459454.98  3.3857 0.0007263 ***
---
Signif. codes:  0 *** 0.001 ** 0.01 * 0.05 . 0.1   1
 
Total Sum of Squares:	2.1218e+16
Residual Sum of Squares: 1.784e+16
R-Squared      :  0.15918
 	Adj. R-Squared :  0.14081
F-statistic: 17.7426 on 18 and 1687 DF, p-value: < 2.22e-16
\end{minted}

\paragraph{}We understand that this is an unbalanced panel due to players having their debut seasons on different seasons in the past decade. Nevertheless, we see some unwanted results here: the more a player plays (\texttt{G} coefficient), the less he is paid. We also see that the better a player’s team performs (\texttt{WL\_percentage} coefficient), the less this player earns. We cannot be satisfied with this regression. Thus, we remove variables that produce these errors and see if the variables that were detected as good predictors are still predictors after our model's transformation:

\begin{minted}[frame=single,linenos,mathescape,fontsize=\small]{r}
Coefficients :
                        Estimate Std. Error t-value  Pr(>|t|)	
Age                    434450.61   30331.37 14.3235 < 2.2e-16 ***
PTS                  	1456.62 	404.42  3.6018 0.0003252 ***
X3P                  	2769.00	3283.94  0.8432 0.3992387	
AST                  	2044.82	1462.27  1.3984 0.1621788	
PosC                   152097.29 2134818.36  0.0712 0.9432104	
PosPF                   69280.69 2098642.23  0.0330 0.9736688	
PosPG                  919394.29 2050932.03  0.4483 0.6540076	
PosSF                 -589410.59 2032567.80 -0.2900 0.7718646	
PosSG                  700798.70 2025538.47  0.3460 0.7293996	
TopPerformerPoints     239969.53 1164789.50  0.2060 0.8368003	
TopPerformerWinShares 4044273.65 1332347.65  3.0354 0.0024384 **
---
Signif. codes:  0 *** 0.001 ** 0.01 * 0.05 . 0.1   1
 
Total Sum of Squares:	2.1218e+16
Residual Sum of Squares: 1.8418e+16
R-Squared      :  0.13194
  	Adj. R-Squared :  0.1172
F-statistic: 23.4073 on 11 and 1694 DF, p-value: < 2.22e-16
\end{minted}

\paragraph{Interpretation}We find in this new regression the same predictors as before. Thus, we can feel disturbed to have removed the "number of games played" variable. Thus, we observe that even if we put this variable in the list of predictors, the regression will also conclude that \texttt{TopPerformerWinShares}, \texttt{Age}, and \texttt{PTS} have significant influence on our outcome variable – salary (i.e. Pr(>\textbart\textbar) < 0.05) – and are probably good predictors of players' salary. We also tried to many other combination of predictors, trying to replace variables by other variables with which they could be correlated (e.g. 2-points scored with total points scored) but we always found that these three variables are the best predictor of players' salaries.
We also tried a random model regression by including the time-invariant variables height and weight but the Hausman test always gave us that the fixed effect model was better.

\paragraph{}In conclusion, we suggest that the most important factor in players' salaries is age of players (which corresponds to their experience) and the number of points they score in one season. As we initially suggested, there is also an exception in player's salaries: players that have the best statistics for win shares in the championship are usually paid 4M\$ than others. In concrete terms, team owner should consider that one year of experience in a player is worth 400k\$. If a player score more points than usual, he could also give bonus up to 2000\$ per point. Finally, owner of the best teams should reward their top performer with a bonus of 4M\$.
